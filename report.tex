%\title{Project Report}
%
%%% Preamble
\documentclass[paper=a4, fontsize=11pt]{scrartcl}
\usepackage[T1]{fontenc}
\usepackage{fourier}
\usepackage{listings}

\usepackage[english]{babel}															% English language/hyphenation
\usepackage[protrusion=true,expansion=true]{microtype}	
\usepackage{amsmath,amsfonts,amsthm} % Math packages
\usepackage[pdftex]{graphicx}	
\usepackage{url}
\usepackage{hyperref}
\usepackage{graphicx}
\usepackage{wrapfig}
\usepackage[margin=1.00in]{geometry}
\usepackage{amsmath}

%%% Custom sectioning
\usepackage{sectsty}
\allsectionsfont{\centering \normalfont\scshape}


%%% Custom headers/footers (fancyhdr package)
\usepackage{fancyhdr}
\pagestyle{fancyplain}
\fancyhead{}											% No page header
\fancyfoot[L]{}											% Empty 
\fancyfoot[C]{}											% Empty
\fancyfoot[R]{\thepage}									% Pagenumbering
\renewcommand{\headrulewidth}{0pt}			% Remove header underlines
\renewcommand{\footrulewidth}{0pt}				% Remove footer underlines
\setlength{\headheight}{3.6pt}
\date{}


%%% Equation and float numbering
\numberwithin{equation}{section}		% Equationnumbering: section.eq#
\numberwithin{figure}{section}			% Figurenumbering: section.fig#
\numberwithin{table}{section}				% Tablenumbering: section.tab#


%%% Maketitle metadata
\newcommand{\horrule}[1]{\rule{\linewidth}{#1}} 	% Horizontal rule

\title{
		\vspace{-1in} 	
		\usefont{OT1}{bch}{b}{n}
		\normalfont \normalsize \textsc{Durham Computer Science} \\ [5pt]
		\horrule{0.5pt} \\[0.4cm]
		\large  Bioinformatics Assignment - LLLL76\\
		\horrule{2pt} \\[0.5cm]
		\vspace{-1in} 	
}

%%% Begin document
\begin{document}
\maketitle
\section*{Question One}

\iffalse
This question is about the Build algorithm from A. V. Aho, Y. Sagiv, T. G. Szymanski, J. D. Ullman. Inferring a tree from lowest common ancestors with an application to the optimization of relational expressions. SIAM Journal on Computing
\fi

\subsection*{A - 25 Marks}

\iffalse
Explain what the algorithm does and how it works in your own words. Do not use pseudocode
\fi

- given a tree \\
- given lineage constraints \\
- in the form (i, j) < (k,l) where (i, j) is the lowest common ancestor of i and j \\
- using this algorithm a tree will be created such that all lineage constraints are observed \\
- if no tree can be created then Null will be returned \\
- this is done by splitting the set of leaf nodes into the number of sets created by removing the root nodes \\
- and observing a set of rules that are to be followed if a tree is to exist \\
- 1 - for each constraint i and j are in the same set \\
- if they are not then the lowest common ancestor is the root node which cannot be a descendent of anything so we have an issue \\
- 2 - for each constraint if k and l are in the same set the i,j,k,l are in the same set \\
- if they are not in the same set then they cannot be compared as they are not in the same descendent line.\\
- 3 - No to leaves are in the same block unless it follows from rules 1 and 2\\


This is done recursively. \\
given a set of non empty leaf nodes and the set constraints\\
if the set contains a single node then return the a tree consisting of only that node \\
if the set contains more than a single node \\
remove the root node of the current set \\
splitting the leaf nodes set into a set of sets \\
if the number of sets is one then return the null tree \\
else \\
for each of the sets \\
restrict the set of constraints to those that contain any relevant leaf nodes \\
calculate the tree that satisfies this altered set of leaf nodes and constraints recursively \\
if the tree returned is the Null tree then return the Null tree \\
if for every subset in cN no null tree is returned then a tree does exist\\

let T be the tree with a new node for its root and whose children are the roots of $T,., 1 =< m -<_ r$ \\
return T

\subsubsection*{B - 20 Marks}

\iffalse
Expand the partition step (given below) in pseudocode
compute πC = S1, S2, . . . Sr;
\fi



\subsubsection*{C - 25 Marks}

\iffalse
Write a recurrence that expresses the running time of Build depending on the
number of different leaf-labels n and the number of constraints m. Use it to
estimate the running time of the algorithm assuming that the partitioning step
runs in time f (n, m) for some function monotonically nondecreasing function f.
\fi

\subsubsection*{D - 25 Marks}

\iffalse
Run the algorithm on the following set of constraints
(e, f) < (k, d) (c, l) < (g, k)
(c, h) < (a, n) (g, b) < (g, i)
(j, n) < (j, l) (g, i) < (d, m)
(c, a) < (f, h) (c, h) < (c, a)
(j, l) < (e, n) (e, f) < (h, l)
(n, l) < (a, f) (j, l) < (j, a)
(d, i) < (k, n) (k, m) < (e, i)
(d, i) < (g, i) (j, n) < (j, f).
You should show the partitioning and the recursive calls at each stage.
\fi

\subsubsection*{E - 30 Marks}

\iffalse
Reverse the Build algorithm, i,e. design an algorithm that takes a tree with labelled leaves
as an input, and produces a set of constraints of the form (i, j) < (k, l), such that when
Build runs on that set, the result is (an isomorphic copy of) the input tree. Prove the
correctness of your algorithm. Also, a smaller output (number of constraints) would give
you a better mark. 
\fi

%%%%%%%%%%%%%%%%%%%%%%%%%%%%%%%%%%%%%%%%%%%%%%%%%%

\section*{Question Two}

\iffalse
This question is about the MinCutSupertree algorithm from  C. Semple and M. Steel. A supertree method for rooted trees. Discrete Applied Mathematics
\fi

\subsubsection*{A - 15 Marks}

\iffalse
One of its properties is that it preserves nesting and subtrees that are shared
by all of the input trees. Point where precisely in the algorithm this property
is achieved.
\fi

\subsubsection*{B - 10 Marks}

\iffalse
Argue that the MinCutSupertree algorithm is a generalisation of the Build
algorithm, i.e. show how to encode a constraint from the inputs of the later as
a tree, which is one of the inputs of the former.
\fi

%%%%%%%%%%%%%%%%%%%%%%%%%%%%%%%%%%%%%%%%%%%%%%%%%%
%%% End document
\end{document}